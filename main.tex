\documentclass{beamer}
% \documentclass[aspectratio=169]{beamer} % utilizar esta opción para relación 16:9
\usepackage[utf8]{inputenc}
\usepackage[spanish]{babel}
\usepackage{booktabs} 
\usepackage{comment}
\usepackage[absolute, overlay]{textpos} 
\usepackage{pgfpages}
\usepackage[font=footnotesize]{caption}
\usepackage{csquotes}
\usepackage{amsmath}
\usepackage{transparent}
\usepackage{textpos}
\usepackage{tikz}
\usepackage{lipsum}

% Definición de colores
\definecolor{gold}{RGB}{254, 206, 0}
\definecolor{bgblue}{RGB}{0, 40, 89}

% Configuración de tema y colores en las diapositivas
\usetheme{Madrid}
\setbeamercolor{title in head/foot}{bg=bgblue}
\setbeamercolor{author in head/foot}{bg=bgblue}
\usecolortheme[named=bgblue]{structure}

% Título y datos de la presentación
\title[Título corto]{Título de la presentación}
\subtitle{Subtítulo}
\institute[]{Banco de Guatemala\\Departamento de Investigaciones Económicas}
\titlegraphic{\includegraphics[height=2.5cm]{banguat.png}}
\author[Rodrigo Chang (BG)]{Rodrigo Chang}
\date{\today}

% Configuración de símbolos de navegación
\addtobeamertemplate{navigation symbols}{}{%
    \usebeamerfont{footline}%
    \usebeamercolor[fg]{footline}%
}

% Configuración de logo
\logo{\transparent{0.4}\includegraphics[scale=0.1]{banguat.png}}

% Configuración de bibliografía
\usepackage[backend=biber,style=authoryear,sorting=nyt]{biblatex}
% Agregar la coma y espacio en las citas con paréntesis (Autor, año)
\renewcommand*{\nameyeardelim}{\addcomma\space}
% Configuración de "et al." cuando son más autores
\DefineBibliographyStrings{spanish}{andothers={et al.}}

% Archivo con referencias bibliográficas
\addbibresource{references.bib}


\begin{document}

% Diapositiva de título
\begin{frame}[plain]
\maketitle
\end{frame}


% Diapositiva de contenidos
\begin{frame}{Contenido}
\tableofcontents
\end{frame}

% Podemos utilizar secciones para separar el contenido de la presentación
\section{Introducción}
\begin{frame}
   En esta presentación utilizamos \textcite{degroot2012probability} para obtener ... 
\end{frame}

\section{Revisión de literatura}
\begin{frame}{Revisión de literatura}
    \lipsum[1]
\end{frame}

\section{Metodología}
\begin{frame}{Metodología}
    \lipsum[1]
\end{frame}

\section{Resultados}
\begin{frame}{Resultados}
    \lipsum[1]
\end{frame}

\section{Conclusiones}
\begin{frame}{Resultados}
    \begin{itemize}
        \item 1
        \item 2
        \item 3
    \end{itemize}
\end{frame}
    
\section{References}
\begin{frame}[allowframebreaks]
    \printbibliography  
\end{frame}


\end{document}